\documentclass{amsbook}
\usepackage{dsfont}
\usepackage[utf8]{inputenc}
\usepackage[italian]{babel}
\usepackage{amssymb}
\usepackage{amsfonts} 
\usepackage{amsthm}
\usepackage{amsmath}
\usepackage{tikz}
\usepackage{xypic}
\usepackage{verbatim}


%Comandi nuovi
\newcommand{\on}{\operatorname}

\newcommand{\QQ}{\mathbb Q}
\newcommand{\ZZ}{\mathbb{Z}}
\newcommand{\CC}{\mathbb{C}}
\newcommand{\RR}{\mathbb R}
\newcommand{\FF}{\mathbb F}
\newcommand{\PP}{\mathbb{P}}
\newcommand{\AAA}{\mathbb{A}}
\newcommand{\HH}{\mathcal{H}}
\newcommand{\OO}{\mathcal{O}}
\newcommand{\Ff}{\mathcal F}


\newcommand{\spec}{\operatorname{Spec}}
\newcommand{\Frac}[1]{\on{Frac}\left( #1 \right)  }
\newcommand{\Pro}{\on{Proj}}
\newcommand{\End}{\on{End}}
\newcommand{\im}{\on{Im}}

\newcommand{\inv}{^{-1}}
\newcommand{\ol}{\overline}

\newcommand{\mm}{\mapsto}

% Ambienti

\newtheorem*{conj}{Congettura}
\newtheorem{teo}{Teorema}
\newtheorem{lem}{Lemma}
\newtheorem{cor}{Corollario}
\newtheorem{prop}{Proposizione}

\newtheorem*{teo*}{Teorema}
\newtheorem*{lem*}{Lemma}
\newtheorem*{cor*}{Corollario}
\newtheorem*{prop*}{Proposizione}

%\theoremstyle{definition}
\newtheorem*{def*}{Definizione}
\newtheorem{ex}{Esercizio}

\theoremstyle{remark}
\newtheorem*{rmk*}{Osservazione}

\begin{document}

\chapter{Definizioni di base}

\begin{def*}[Ordine di annullamento di una funzione meromorfa]
Data $X$ varietà algebrica (=schema integro e ti tipo finito su un campo) e $V$ sottovarietà di codimensione $1$ e $f \in \OO_{X,V}$ una funzione regolare lungo $V$ definiamo $\on{ord}_V(f)=l(\OO_V / f\OO_V)$. Possiamo poi estendere questa definizione a funzioni meromorfe su $X$ 
\end{def*}

Proprietà ed osservazioni:
\begin{itemize}
\item Grazie al fatto che $X$ è una varietà, questa è una buona definizione in quanto $\OO_V$ è un anello integro di dimensione $1$, dunque si può calcolare la lunghezza di tutti i suoi quozienti 
\item L'ordine di annullamento è additivo: $\on{ord}_V(fg)=\on{ord}_V(f)+\on{ord}_V(g)$  e questo permette di estendere la definizione alle funzioni meromorfe
\item Se $\OO_V$ è un anello di valutazione con parametro uniformizzante $t$ allora per calcolare l'ordine di annullamento di $f$ basta confrontare $f$ con una potenza di $t$ (in particolare possiamo fare sempre questa cosa se $X$ è regular in codimension 1, oppure se $X$ è normale)
\item Se $f,g$ sono polinomi in due variabili e $f$ è irriducibile, allora per ogni punto $P \in V(f)\subset \AAA^2$  si ha che l'ordine di annullamento di $g(\pmod f)$ in $P$ è la molteplicità di intersezione di $f$ e $g$ in $P$.
\item Data $X$ non normale con normalizzazione $\pi:\widetilde{X}\mm X$, si ha che 
$$\on{ord}_V(f)= \sum_{V_i \in \pi\inv V}[k(V_i):k(V)]\on{ord}_{V_i}(f)$$
Inoltre questo ci permette di ricondurci sempre al caso normale.
\item Non vale una disuguaglianza ultrametrica (esempio sulla cubica nodata)
\end{itemize}

Per dimostrare il comportamento dell'ordine di annullamento sotto normalizzazione abbiamo usato che 
\begin{lem}
Se $B$ è un anello noetheriano con massimali $q_1,\dots,q_r$ allora 
$$ B= \prod B_{q_i}$$
\end{lem}

\begin{lem}
Se $A$ è un anello di dimensione $1$ e $B$ è un'estensione "finita" (nel senso che $l(B/A)<\infty$) allora $l(A/fA)=l(B/fB)$
\end{lem}

Una volta definito cos'è il divisore di una funzione meromorfa possiamo dare degli invarianti agli schemi algebrici su un un campo $k$ (ovvero schemi noetheriani e di tipo finito su un campo) che siano degli analoghi dei gruppi di omologia. 

\begin{def*}
Data $X$ uno schema algebrico su un campo $k$ il gruppo dei cicli $Z_mX$ è il gruppo libero generato dalle sotto-varietà di $X$ (ovvero schemi integri su $k$ immersi in $X$) di dimensione $m$. Indichiamo poi con $A_m(X)$ l'$m$-esimo gruppo di Chow di $X$ è il quoziente di $Z_mX$ per il sottogruppo generato dai $\div \phi$ al variare di $\phi $ funzione razionale definita su una $m+1$-sottovarietà di $X$
\end{def*}

In particolare possiamo associare ad ogni sottochema chiuso di $X$ un ciclo:
\begin{def*}
Dato $Y\subset X$ sottoschema chiuso con componenti irriducibili $Y_1, \dots Y_r$ (a cui diamo la struttura ridotta) definiamo $$ [Y]= \sum l(O_{Y,Y_i}) Y_i=\sum m_i Y_i$$
Dove $m_i$ (detta molteplicità geometrica di $Y_i$ in $Y$) si guarda nell'anello locale del punto generico di $Y$
\end{def*}

In particolare con l'ultima definizione si può interpretare in un altro modo il divisore di una funzione meromorfa:
\emph{Se} $f:V\mm \PP^1$ è un morfismo dominante definito su una varietà, allora possiamo interpretarlo come funzione meromorfa e si ha che $\div f= [f\inv(0)] - [f\inv (\infty)]$ dove con $f\inv(0)$ è la fibra di zero con la struttura di sottoschema chiuso dato dal prodotto fibrato.

Ora vorremmo spostare cicli con mappe
\begin{def*}
Data $f:X\mm Y$ mappa propria allora definiamo $(f_*)_m:Z_m(X)\mm Z_m(Y)$ tramite la seguente definizione sui generatori:
$$ f_*(V)=[k(V):k(W)] W$$ se $W=f(V)$ (intesa come immagine set-theoretic a cui diamo la struttura ridotta) ha dimensione  $m$ come $V$
$f_*(V)=0$ se $W=f(V)$ ha dimensione minore di $m$
\end{def*}

\begin{def*}
Data $f:X\mm Y$ una mappa piatta di dimensione relativa costante $d$ definimo il pull-back di cicli $(f^*)_m:Z_m(Y)\mm Z_{m+d}(X)$ come la mappa che agisce nel seguente modo sui generatori: $f^*(V)=[f\inv V]$
\end{def*}

Tra le prime cose da dimostrare ci sono essnzialmente due cose:
\begin{itemize}
\item La funtorialità di queste mappe (cosa succede componendo mappe proprie o mappe piatte), che è vera già al livello di cicli (cosa interessante di per sé). Questa si dimostra facilmente nel caso del push-forward basandosi sulla moltiplicatività del grado delle estensioni di campo, mentre per il flat pull-back si tratta di dimostrare una proposizione a parte, che si riconduce ad un calcolo locale e ad un lemma di algebra sulle dimensioni  di anelli artiniani, ovvero 
\begin{lem}
Dato $Z\subset Y$ immersione chiusa $f^*[Z]=[f\inv Z]$
\end{lem}
\noindent Il lemma algebrico, invece è
\begin{lem}
Data $A_P\mm B_Q$ estensione di anelli locali artiniani, allora
\[ l(B)=l_B(B)=l_A(A)\cdot l_B(B/PB)=l(A)l_B(B\otimes k(P))	\]
\end{lem}

\item Una relazione di commutatività fra il flat pull-back e il proper push-forward, che però è vera solo per diagrammi cartesiani, ovvero la seguente regola: Se nel seguente diagramma cartesiano $f$ è propria e $g$ è piatta, allora $f'_*\circ g'^*= g^*f_*$
%riscrivere il grafico con xy o un altro pacchetto
$$\xymatrix{
X' \ar[rr] \ar[dd] &       & Y'\\
& \square \\
X\ar[rr] & &Y\\
}$$
Questa relazione si riconduce al seguente lemma algebrico:
\begin{lem}
Data $A_P\mm B_Q$ estensione di anelli locali artiniani e dato $M$ un $B$-modulo f.g. allora 
\[ l_A(M)=[k(Q):k(P)]l_B(M) \] 
\end{lem}

\item Sia il flat pull back che il proper push-forward preservano l'equivalenza razionale. In realtà si può dire molto di più in quanto possiamo anche ricostruire le relazioni al livello di cicli. 
\end{itemize}


\end{document}